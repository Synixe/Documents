\documentclass[10pt,a4paper]{article}
\usepackage[utf8]{inputenc}
\usepackage{amsmath}
\usepackage{amsfonts}
\usepackage{amssymb}

\title{Constitution of the Synixe Gaming Community}
\date{Effective March 01, 2018}

\begin{document}
\maketitle
\newpage
\section{NAME}
\paragraph{}
The name of this organization shall be Synixe Gaming Community, in shorthand and hereby refered to as Synixe.
\section{PURPOSE}
\paragraph{}
The purpose of Synixe is to provide a relaxed, fair and organized environment for a community of gamers. This constitution shall be used to keep the organization commited to that goal and shall not be changed in such a way that compromises that core goal.
\section{DEFINITIONS}
\paragraph{Rule} A explicit regulation or principale governing the conduct of the organization's members.
\paragraph{Resolution} A decision made by the Executive Branch regarding a member's conduct or the breaching of an organization rule and the resulting action taken upon them.
\paragraph{Proposal} A proposed addition, change or removal to an organization' rules.
\paragraph{Constitutional Ammendment} A change to the Constitution of the Synixe Gaming Community.
\paragraph{Confirming} When a constitutional ammendment is voted on in a session of the Legislative Branch it is "confirmed", allowing a Joint Session to then vote on if it shall be passed.
\paragraph{Meeting} An assembly of members of a branch that cannot result in a change to the organization's rules or constitution.
\paragraph{Session} An assembly of members of the Legislative Branch that can vote on proposals and confirm constitutional ammendments.
\paragraph{Joint Session} An assembly of members of both branchs that can vote to pass a constitutional ammendment.
\section{MEMBERSHIP}
\subsection{Gaining Mebership}
\paragraph{}
Synixe shall permit membership to anyone that wishes to join that is at least 13 years of age. There is no limit to the number of members the organization can have and there is no limit in the duration a person may be a member of the organization. Upon a person joining Synixe they are given the designation of "New" in the organization's system for designations.
\subsection{Activity}
\paragraph{}
Once a member of the organization participates in an organization sanctioned event they will be given the designation of "Active". After 3 sanctioned events without attendance they will be given the designation of "Inactive" and have the designation of "Active" removed.
This designation can be reveresed by attending a sanctioned event.
\subsection{Leaving}
\paragraph{}
A member may leave Synixe at anytime unless they are a member of the Legislative or Executive Branch.
\subsection{Removal} \label{member-removal}
\subsubsection{Inactivity}
\paragraph{}
After a period of 30 days after receiving the designation of "Inactive" a member is to be sent an offer of re-engagment by a member of the Executive Branch of the organization. After 15 days of either no response or a decline to the offer of re-engagment the person's status as a member of the organization is terminated.
\subsubsection{Misconduct} \label{misconduct}
\paragraph{}
A member may be removed by a resolution of the Executive Branch if they have breached a rule or are engaging in repeated desruptive actions.
\subsection{Communication With Former Members} \label{communication-former-members}
\paragraph{}
 After a person is no longer a member of Synixe they are to be removed from any social platforms they were using to engage formally with the organization. Members of the organization are free to continue communication with a former member of Synixe without action being taken against them by the Executive Branch unless such communication is relayed to a public social platform shared with the community after a direct request to cease such actions. If a member continues to relay information a resolution can be made by the Executive Branch to terminate that person's membership with the organization.
\section{ORGANIZATION}
\subsection{Branches}
\paragraph{}
The organization shall have two branches of governance: Executive and Legislative.
All members of the Legislative branch are also members of the Executive branch. Members of the Legislative branch shall never hold a majority in the Executive branch.
\subsection{Legislative Branch}
\subsubsection{Definitions}
\paragraph{}
The members of the Legislative Branch shall be known as a "Manager" and be given this designation in the organization's system for designations.
\subsubsection{Powers}
\paragraph{}
The members of the Legislative Branch have the ability to create new proposals and constitutional ammendments.
\subsubsection{Responsibilies}
\paragraph{}
The Legislative Branch is responsible for ensuring the constitution is properly followed and enforced.
\subsubsection{Term Length}
\paragraph{}
A member of the Legislative Branch may serve their position indefinitely.
\subsubsection{Resignation}
\paragraph{}
A member of the Legislative Branch may declare their resignation at any time to the Legislative Branch. They must continue serving their position for a period of 14 days. During this period the Legislative Branch will select a member of the Executive Branch to become a member of the Legislative Branch. The candidate's confirmation vote will take place at the next session of the Legislative Branch. If the Legislative Branch is tied in their selection of a new member, the Executive Branch will vote on the candidates involed in the tie and will cast their vote as one in the Legislative Branch's selection process.
\subsection{Executive Branch}
\subsubsection{Definitions}
\paragraph{}
The members of the Executive Branch shall be known as a "Moderator" and be given this designation in the organization's system for designations.
\subsubsection{Size}
\paragraph{}
The size of the Executive Branch shall never exceed 10\% of the total members of Synixe or fall below 3\%.
\subsubsection{Powers}
\paragraph{}
The members of the Executive Branch have the power to enforce the rules of Synixe via resolutions.
\subsubsection{Responsibilies}
\paragraph{}
The members of the Executive Branch are responsible for enforcing the constitution and the organization's rules.
\subsubsection{Term Length}
\paragraph{}
A member of the Executive Branch may serve their position indefinitely unless removed.
\subsubsection{Resignation}
\paragraph{}
A member of the Executive Branch may declare their resignation at any time to the Legislative Branch. They must continue serving their position for a period of 7 days. During this period the Legislative Branch will select a member of the organization to become a member of the Executive Branch. If the Legislative Branch is tied in their selection of a new member, the Executive Branch will vote on the candidates involed in the tie and will cast their vote as one in the Legislative Branch's selection process.
\subsubsection{Removal}
\paragraph{}
A member of the Executive Branch may be removed from their position by a two thirds vote during a session of the Legislative Branch if they have failed to complete the responsibilites of their position.
\paragraph{}
When a member of the Executive Branch is removed from their position they do not need to be removed from Synixe as per Section \ref{communication-former-members} unless applicable under Section \ref{misconduct}.
\section{MEETINGS \& SESSIONS}
\subsection{Meetings \& Sessions of the Legislative Branch}
\subsubsection{Meetings}
\paragraph{}
The Legislative Branch can call a meeting at any time if a majority of the members of the Legislative Branch are present. During meetings the Legislative Branch can put forth proposals to be voted on at the next session of the Legislative Branch. During meetings the Legislative Branch can put forth constitutional ammendments to be confirmed at the next session of the Legislative Branch. Members of the Executive Branch are not allowed to observe meetings of the Legislative Branch.
\subsubsection{Sessions}
\paragraph{}
Sessions of the Legislative Branch must be announced to the Executive Branch at least 48 hours prior to the start time of the session. All members of Synixe are allowed to observe sessions of the Legislative Branch.
\paragraph{}
In order for a session to take place a majority of the members of the Legislative Branch are required to be present.
\subsection{Meetings of the Executive Branch}
\paragraph{}
The Executive Branch can call a meeting at any time if at least two members are present. Only members of the Executive Branch may be present.
\subsection{Joint Sessions}
\paragraph{}
Members of the Legislative Branch can call for a Joint Session after a constitutional ammendment has been confirmed at a session of the Legislative Branch. All members of the Legislative Branch must be present and a majority of the members of the Executive Branch must be present.
\subsection{Voting}
\subsubsection{Resolution}
\paragraph{}
At a meeting of the Executive Branch members of the Executive Branch can vote to remove a member of the organization that is not a member of the Legislative or Executive Branch as allowed under section \ref{member-removal}. A majority of the members present voting in favour is required to pass a resolution.
\subsubsection{Proposals}
\paragraph{}
Voting on proposals will take place at sessions of the Legislative Branch. A majority is required to pass any proposal.
\subsubsection{Constitutional Ammendments}
\paragraph{}
Voting on confirming Constitutional Ammendments can occur when two thirds of the Legislative Branch are present at a session of the Legislative Branch. Two thirds of the Legislative Branch is required to confirm any Constitutional Ammendment.
\paragraph{}
Once confirmed the Constitutional Ammendment will be voted on in a joint session. Two thirds of the member's of both branches in favour is required to pass a constitutional ammendment. The ammendment will come into effect after a period of 3 days of the vote.

\end{document}