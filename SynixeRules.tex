\documentclass[10pt,a4paper]{article}
\usepackage[utf8]{inputenc}
\usepackage{amsmath}
\usepackage{amsfonts}
\usepackage{amssymb}

\title{Rules of the Synixe Gaming Community}

\begin{document}
\maketitle
\newpage

\section{General Operations}
\subsection{Platforms}
\begin{enumerate}
	\item Discord is to be the primary form of text communication.
	\item TeamSpeak 3 is to be the primary form of voice communications.
	\item Discord Roles will be used as the organization's system for designations.
	\item All designations must be also shown in the TeamSpeak server.
\end{enumerate}
\subsection{Servers}
\begin{enumerate}
	\item All dedicated servers used by the community will be hosted by a member of the Executive Branch.
	\item The TeamSpeak server IP address shall not be shared until required.
	\item If the TeamSpeak server is unavailable temporary channels shall be created in Discord.
\end{enumerate}
\section{Age}
\paragraph{}
The minimum age required to join is 13. New members between the ages of 13-15 inclusive will be subject to a 3 month probabtion period during which they can be removed if they can not integrate into the community smoothly.
\section{Content} \label{content}
\subsection{Moderators}
\paragraph{}
Moderators are allowed to remove any content they feel is in objective breach of any definitions under Section \ref{content}.
\subsection{Inappropriate Content}
\begin{enumerate}
	\item Any content that is pornographic in nature will be removed.
	\item Any content containing horror, cruelty or violence at the discretion of the Moderation Team will be removed.
	\item Any content containing glorification of criminal activities at the discretion of the Moderation Team will be removed.
\end{enumerate}
\subsection{Harassment}
\begin{enumerate}
	\item Targeting an individual because of their race or ethnicity will not be tolerated.
	\item Targeting an individual because they practive a specific religion will not be tolerated.
	\item Content, symbols, or behaviours of a sexual nature that make the target feel uncomfortable will not be tolerated.
\end{enumerate}
\section{Cheating \& Exploiting} \label{cheating}
\subsection{Gaming}
\begin{enumerate}
	\item Any deliberate action or attempt to enhance your game play or to give yourself an unfair advantage will not be tolerated.
	\item Any modification of game files or additional modifications outside of what is required by Synixe will not be tolerated.
\end{enumerate}
\subsection{Exploiting}
\begin{enumerate}
	\item Any attempt to gain access to administrative tools will result in an immediate permanent ban.
\end{enumerate}
\section{Dual-Clanning} \label{dual-clan}
\begin{enumerate}
	\item Members of Synixe that hold a role of Manager or Moderator are asked to refrain from holding a leadership role in other communities. If they wish to hold a leadership position in another community we ask that they resign from their position in Synixe.
	\item All members without the designation of Manager or Moderator are free to play in any other communities.
\end{enumerate}
\section{Conduct} \label{conduct}
\paragraph{}
All members of Synixe are expected to treat their fellow members in a considerate way at all times. Disagreements are fine, hostility is not.
\subsection{Disruptive Behaviour}
\paragraph{}
Disruptive behaviour is any action that degrades the quality of the community for other members at the discrestion of the Moderation Team.
\section{Reporting}
\paragraph{}
If you know about anyone breaking a rule found in Sections \ref{content}, \ref{cheating}, \ref{dual-clan}, or \ref{conduct} please report it to a moderator of Synixe in a direct message or by using the ?anon command in a direct message with the Synixe Bot.
\section{Names}
\begin{enumerate}
	\item Names can not be offensive in any way and are subject to all rules under Section \ref{content}.
	\item Names should be clear and easy to read.
	\item Names can not exceed 4 syllables.
	\item Names should be easily pronouncable so people can easily address you on voice comms.
	\item Names must be consistent accross all platforms. This helps people learn who is who.
\end{enumerate}
\end{document}